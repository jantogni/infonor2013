\section{Primeros pasos del Chilean Virtual Observatory}

En el Chilean Virtual Observatory (ChiVO), se han generado instancias y colaboraciones
para llevar a cabo este ambicioso proyecto. En estas, se han clarificado las
posibles limitaciones y necesidades que tendrá la comunidad astronómica en
el mediano plazo con las cuales se han planteado objetivos del sistema. Esta
sección abordará los objetivos del sistema y la iniciativa que está desarrollando ChiVO.

\subsection{Objetivos del sistema}

Los volúmenes de datos a gran escala que generan y generarán los observatorios
astronómicos actuales y futuros en Chile, han suscitado nuevas necesidades que
permiten el desarrollo de nuevas herramientas y técnicas de análisis de datos.

Para tener una idea general de la cantidad de datos que requerirán ser
clasificados y procesados, se puede tomar en cuenta el proyecto ALMA,
inaugurado en marzo del 2013, en el cual se estima que serán generados más de 1
TB de datos diarios en su máxima capacidad operativa.
%en el cual cuando se encuentre completamente operativo
%(con todo su conjunto de antenas), se generarán más de 1 TB de datos por día de
%observación.

La manipulación de altos volúmenes de datos generan complicaciones en diversos
aspectos:
%\begin{itemize}
	%\item Almacenamiento, es necesario tener un centro de procesamiento de datos con
	\textbf{Almacenamiento}, es necesario tener un centro de procesamiento de datos con
		la capacidad de almacenamiento suficiente y acorde a las necesidades de consumo
		de dichos datos, sin dejar de lado el espacio físico que dichos equipos
		necesitarán y la arquitectura detrás del almacenamiento en sí. \\
	\textbf{Acceso}, se deben establecer mecánicas y normativas de accesos para
		cualquier persona, ya sean astrónomos o no, lo cual requiere de un sistema que
                permita acceder a él desde cualquier lugar. En este proyecto
		se ha decidido por sistema web, ya que es un mecanismo adecuado para suplir la
                presente necesidad. \\
	\textbf{Procesamiento}, el procesamiento de datos es un área particular de la
		informática, la cual conlleva a entender tanto la naturaleza como la estructura
		de los datos como también las herramientas y técnicas que se pueden utilizar
		para llevarlo a cabo. Al tener grandes volúmenes de datos, el procesamiento a
		realizar, ya sean correcciones, calibraciones, análisis, etc., exigen más tiempo
		del habitual y por supuesto que más recursos computacionales, elementos que un
		usuario no posee. Es por lo anterior que un cluster computacional puede ser una
		herramienta útil para realizar esta labor. \\
                %Por lo que un cluster computacional, puede ser una herramienta útil para
                %facilitar esta labor.
%\end{itemize}

Los anteriormente detallado pertenece a las motivaciones principales
por las cuales actualmente en Chile se está desarrollando un proyecto que
busca crear una plataforma de procesamiento de datos de gran escala, el cual
compromete como uno de sus entregables un Observatorio Virtual del que se
espera que garantice rapidez y eficiencia tanto al acceso de información
existente y servicios astronómicos como en el análisis de dicha información.

\subsection{Iniciativa que lo desarrolla}
Actualmente se está desarrollando el proyecto titulado ``Desarrollo de una
plataforma astroinformática para la administración y análisis de datos de gran
escala'', financiado con fondos gubernamentales (FONDEF D11I1060) con duración
de 28 meses en el cual participan las siguientes instituciones:
\begin{itemize}
	\item Atacama Large Milimeter/submilimeter Array
	\item Consorcio Red Universitaria Nacional (Reuna)
	\item Universidad Técnica Federico Santa María
	\item Universidad de Chile
	\item Universidad Católica de Chile
	\item Universidad de Concepción
	\item Universidad de Santiago de Chile
\end{itemize}

Los objetivos del proyecto están relacionados con el diseño e implementación de
un observatorio virtual, el cual deberá cumplir con los estándares de IVOA.
Además, los astrónomos investigadores de este, crearán instancias donde
presentarán problemáticas que enfrentan como comunidad ante el procesamiento de
datos que serán resueltos mediante técnicas computacionales conocidas por los
investigadores del área de computación.

El presente proyecto se realiza en estrecha colaboración con ALMA, quienes
aportan su visión desde el punto de vista de observatorio, comparten
conocimiento respecto a los modelos y tipos de datos que se usan, y además se
establecerán políticas de colaboración para facilitar el acceso a los datos.

Por otro lado, el Consorcio Red Universitaria Nacional Reuna (REUNA) y el
National Laboratory for High Performance Computing (NLHPC) juegan un rol
importante, ya que una de las problemáticas a resolver es la conectividad de
altas tasas de transmisión de datos (REUNA) y el almacenamiento de datos que
exigen grandes capacidades para ello (NLHPC).

En conjunto estas instituciones unen esfuerzos para lograr crear y establecer
en el tiempo una plataforma sin precedente en el área astroinformática Chilena,
el Chilean Virtual Observatory\footnote{ChiVO \url{http://www.chivo.cl/}}.
