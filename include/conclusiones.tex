\section{Conclusiones y trabajo futuro}

El observatorio virtual es un \textit{framework} que le permite a los astrónomos y
comunidad en general buscar en múltiples servidores de datos de forma
transparente, pero un foco de mayor interés para la comunidad informática es
que guía la construcción de un sistema robusto a partir de tecnologías,
estándares y protocolos unificados. Esto enmarcado en su arquitectura orientada
a la intercomunicación mediante 3 capas: usuarios, intermedia (\textit{virtual
observatory}) y de recursos, y que gracias a la especificación de cada una,
están formalizados los formatos de representación de datos y los métodos por
los cuales se puede acceder a los mismos.

En el mundo hay 19 observatorios virtuales miembros de IVOA, y próximamente
Chile se unirá a esta organización mediante el Chilean Virtual Observatory, una
iniciativa en colaboración con 5 Universidades Chilenas, ALMA y REUNA.
Inicialmente el proyecto está centrado en capturar los requerimientos de los
astrónomos para una plataforma de este tipo y también en ver el modo de cómo
modelar los datos provenientes de ALMA de tal manera que sean útiles y
accesibles para la comunidad en general. Paralelamente se está haciendo un
estudio acabado de la arquitectura que plantea IVOA para usar y respetar
sus estándares de desarrollo y así asegurar la interoperabilidad de ChiVO
con los otros VOs del mundo.
